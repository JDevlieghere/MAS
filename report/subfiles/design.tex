\documentclass[../main.tex]{subfiles}
\begin{document}
\subsection{Agent interaction}
Two forms of interaction are implemented between agents.
Both forms make use of the beacons placed either on the parcels or the trucks and a messaging system.
Beacons provide a ping interface that replies true only the first time when it is pinged (after a reset).
Interactions require a master role that is assigned to the agent depending on the reply of the ping.
Another important note is that regular messages (not pings) are sent at the end of each tick.
Which means a message sent at agent A at $t=n$ is received at agent B at $t=n+1$.
If agent B replies ASAP the reply is received by agent A at $t=n+2$.
This implies agent A has to wait a full tick before a reply can be expected.
Both interactions are statefull.

\subsubsection{Auctions}
Our auction system is a one-shot sealed bid auction which is started as soon as a parcel becomes available.
The bids are dependant ont two factors:
\begin{enumerate}
	\item The amount of parcels the truck has picked up
	\item The amount of planned pickups of a truck
\end{enumerate}
In this case a lower bid wins over a higher one since it implies the agent has less work to do.
When an agent discovers a parcel, meaning the agent is within it's beacon radius, it is pinged.
If the reply is true the agent becomes the auctioneer else another agent pinged first and has already received the auctioneer status. 
Below is a chronological description of the auction protocol:

\paragraph{Tick 1: UNAUCTIONED}
The autctioneer broadcasts a message to all agents (infinite communication radius) that it wants to auction a certain parcel \texttt{P}.
During this tick all agents within range of the parcel \texttt{P} should have discovered and thus pinged the parcel.
\paragraph{Tick 2: PENDING}
The auctioneer waits one tick for the messages to arrive and replies to be sent.
Other agents receive the auction participation request for parcel \texttt{P} and check whether they have discovered that same parcel. 
If this is the case they reply with their bid to the acutioneer.
\paragraph{Tick 3: AUCTIONING}
The auctioneer receives all bids from participating agents.
It now simply extracts the best bid from all bids (including it's own bid).
If the auctioneer wins it enqueues the parcel for pickup. 
If another agent wins the auctioneer sends an assignment message to the winning agent.
\paragraph{[Optional] Tick 4: ASSIGNMENT}
The winning agent receives it's assiginment message and enqueues the parcel \texttt{P} for pickup.
All other agents receive nothing and carry on with their other tasks 

\subsubsection{Exchanges}
As soon as two agents are within eachothers beacon range the protocol is started.
The first agent to send a ping message to the other agent becomes the master of the exchange.
An already pinged agent won't try to ping the other agent and becomes the slave of the exchange.

\paragraph{Tick 1:MASTER INITIATE}
The master agent received it's status by being the first agent to ping the other agent.
The master sends an exchange request message to the slave agent. 
\paragraph{Tick 2:MASTER PENDING}
The master waits for request to be received and replied to be send.
\paragraph{Tick 3:MASTER PLANNING}
The master analyzes the reply to decide whether an exchange could be beneficial.
Now two options should be considered:
Option 1: Exchange is not beneficial.
This can be the case if each agent has exactly one parcel or if there is only one parcel for the two agents. 
In this case an empty assignment message is sent and the master sets it's exchange status on resetting.
Option 2: Exchange is beneficial.
In any other case both agents could benefit from exchanging parcels.
The master runs a k-means clustering on all the parcels to create two groups with delivery locations clustered.
When this is done an assignment message is sent with the meeting location which is halfway between both agents.
The master sets it's exchange status to MEETING.
\paragraph{Tick 4-n:MASTER MEETING}
The master drives to the location it decided in the PLANNING stage and waits till the slave arrives (which shouldn't be long). 

\paragraph{Tick n+1:MASTER EXCHANGING}
The master executes the exchange as it was planned in the PLANNIG stage. 
When the exchange is sucessfully completed it sets it's status to RESETTING.
\paragraph{Tick 4/n+2:MASTER RESETTING}
This case is for when either the exchange is succesfully completed or stopped permaturely.
The master checks as soon as the other truck is out of range and then resets it's beacon and exchange status.

\paragraph{Tick 1-2:SLAVE INITIATE}
The slave checks for incoming exchange request messages.
\paragraph{Tick 3:SLAVE PENDING}
The slave waits for reply to be received and for further instructions to be send.
\paragraph{Tick 4:SLAVE PLANNING}
The slave checks for incoming assignment messages. 
Again two options are possible.

Option 1: Empty assignment message.
This means the exchange will not take place.
The slave sets it's status to RESET.

Option 2: Assignment with meeting point.
This means an exchange will take place at the given meeting point.
The slave sets it's status to MEETING.
\paragraph{Tick 5-n:SLAVE MEETING}
The slave drives to the location it decided in the PLANNING stage and waits till the master arrives (which shouldn't be long). 
\paragraph{Tick n+1:SLAVE EXCHANGING}
The slave participates in the exchange.
\paragraph{Tick 4/n+2:SLAVE RESETTING}
The slave checks whether the master agent is out of range an then resets it's beacon and exchange status.
\end{document}
