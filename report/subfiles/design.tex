\documentclass[../main.tex]{subfiles}
\begin{document}

Two forms of interaction are implemented between agents.
Both forms make use of the beacons placed either on the parcels or the trucks and a messaging system.
Beacons provide a ping interface that replies true only the first time when it is pinged (after a reset).
Interactions require a master role that is assigned to the agent depending on the reply of the ping.
Another important note is that regular messages (not pings) are sent at the end of each tick.
Which means a message sent at agent A at $t=n$ is received at agent B at $t=n+1$.
If agent B replies ASAP the reply is received by agent A at $t=n+2$.
This implies agent A has to wait a full tick before a reply can be expected.
Both interactions are statefull.

\subsubsection{Auctions}

Our auction system is a one-shot sealed bid auction which is started as soon as a parcel becomes available.
The bids are dependant ont two factors:
\begin{enumerate}
	\item The amount of parcels the truck has picked up
	\item The amount of planned pickups of a truck
\end{enumerate}
In this case a lower bid wins over a higher one since it implies the agent has less work to do.
When an agent discovers a parcel, meaning the agent is within it's beacon radius, it is pinged.
If the reply is true the agent becomes the auctioneer else another agent pinged first and has already received the auctioneer status.
Below is a chronological description of the auction protocol:

\paragraph{Tick 1: UNAUCTIONED}
The autctioneer broadcasts a message to all agents (infinite communication radius) that it wants to auction a certain parcel \texttt{P}.
During this tick all agents within range of the parcel \texttt{P} should have discovered and thus pinged the parcel.
\paragraph{Tick 2: PENDING}
The auctioneer waits one tick for the messages to arrive and replies to be sent.
Other agents receive the auction participation request for parcel \texttt{P} and check whether they have discovered that same parcel.
If this is the case they reply with their bid to the acutioneer.
\paragraph{Tick 3: AUCTIONING}
The auctioneer receives all bids from participating agents.
It now simply extracts the best bid from all bids (including it's own bid).
If the auctioneer wins it enqueues the parcel for pickup.
If another agent wins the auctioneer sends an assignment message to the winning agent.
\paragraph{[Optional] Tick 4: ASSIGNMENT}
The winning agent receives it's assiginment message and enqueues the parcel \texttt{P} for pickup.
All other agents receive nothing and carry on with their other tasks

\subsubsection{Exchanges}

As soon as two agents are within eachothers beacon range the protocol is started.
The first agent to send a ping message to the other agent becomes the master of the exchange.
An already pinged agent won't try to ping the other agent and becomes the slave of the exchange.

\paragraph{Master exchange flow}
\begin{description}
	\item [Tick 1:MASTER INITIATE]
	The master agent received it's status by being the first agent to ping the other agent.
	The master sends an exchange request message to the slave agent.
	\item[Tick 2:MASTER PENDING]
	The master waits for request to be received and replied to be send.
	\item[Tick 3:MASTER PLANNING]
	The master analyzes the reply to decide whether an exchange could be beneficial.
	Now two options should be considered: \\
	\textbf{Option 1}: Exchange is not beneficial.
	This can be the case if each agent has exactly one parcel or if there is only one parcel for the two agents.
	In this case an empty assignment message is sent and the master sets it's exchange status on resetting.
	\\
	\textbf{Option 2}: Exchange is beneficial.
	In any other case both agents could benefit from exchanging parcels.
	The master runs a k-means clustering on all the parcels to create two groups with delivery locations clustered.
	When this is done an assignment message is sent with the meeting location which is halfway between both agents.
	The master sets it's exchange status to MEETING.
	\item[Tick 4-n:MASTER MEETING]
	The master drives to the location it decided in the PLANNING stage and waits till the slave arrives (which shouldn't be long).
	\item[Tick n+1:MASTER EXCHANGING]
	The master executes the exchange as it was planned in the PLANNIG stage.
	When the exchange is sucessfully completed it sets it's status to RESETTING.
	\item[Tick 4/n+2:MASTER RESETTING] 
	This case is for when either the exchange is succesfully completed or stopped permaturely.
	The master checks as soon as the other truck is out of range and then resets it's beacon and exchange status.
\end{description}

\paragraph{Slave exchange flow}
\begin{description}
	\item[Tick 1-2:SLAVE INITIATE]
	The slave checks for incoming exchange request messages.
	\item[Tick 3:SLAVE PENDING]
	The slave waits for reply to be received and for further instructions to be send.
	\item[Tick 4:SLAVE PLANNING]
	The slave checks for incoming assignment messages.
	Again two options are possible.
	\\
	\textbf{Option 1}: Empty assignment message.
	This means the exchange will not take place.
	The slave sets it's status to RESET.
	\\
	\textbf{Option 2}: Assignment with meeting point.
	This means an exchange will take place at the given meeting point.
	The slave sets it's status to MEETING.
	\item[Tick 5-n:SLAVE MEETING]
	The slave drives to the location it decided in the PLANNING stage and waits till the master arrives (which shouldn't be long).
	\item[Tick n+1:SLAVE EXCHANGING] 
	The slave participates in the exchange.
	\item[Tick 4/n+2:SLAVE RESETTING]
	The slave checks whether the master agent is out of range an then resets it's beacon and exchange status.
\end{description}

\subsection{Activities and their order}
All tasks and decisions of the agents are made or executed by certain activities.
The order of these activities influences the entire system. 
\begin{enumerate}
	\item PickupActivity: This is the activity of loading the truck with a parcel when the truck is standing on the pickup location of the parcel.
	\item DeliverActivity: This is the activity of unloading the truck with a parcel at it's delivery location. 
	\item ExchangeActivity: This activity starts or continues the exchange protocol between two agents. 
	\item FetchActivity: This activity is used for going to and determining the order of pickup locations. 
	\item TransportActivity: This activity is used for going to and determining the order of delivery locations.   
	\item AuctionActivity: Participating or starting an auctiong for newly discovered parcels.
	\item DiscoverActivity: Discover parcels which are emitting their beacon signal. 
	\item ExploreActivity: Explore the world in search for new parcels.  
\end{enumerate}
PickupActivity and DeliverActivity are obviously executed first since all other activities lead up to this two activities. 
They are executed soley if the truck is on either a pickup or delivery location of a parcel it should picklup or deliver.
Secondly the exchange activity is executed because we want to avoid missing a chance to improve the performance two nearby trucks.
After that Fetching and Transporting is our main concern.
Because auctioning is based on how occupied the agent currently is auctioning is only allowed when the agent is not actively fetching or transporting a parcel.
Discovery is executed after auctioning because discovery of another parcel could disturb an auction that is in progress.
Lastly exporation is executed. This has ofcourse (even when enabled) the lowest priority.
Only when no other activity being exectuted active exploration could be useful.

\subsection{Scheduling strategies}
\subsubsection{Pickup Strategy}
In the FetchActivity two possible scheduling startegies were used.
One naive FIFOPickupStrategy which was mainly used for testing.
And a second NearestPickupFirstStrategy which is used in all our real experiments.
As the name suggest this strategy reschedules pickups depending on the distance from current location of the truck to the parcels to pickup.

\subsubsection{Delivery Strategy}
A number of different delivery strategies were implemented and tested. 

The idea for the ClusterDeliveryStrategy is that we try to cluster the parcels by there delivery locations. 
Using K-means clustering deliverable parcels are being grouped by their delivery location.
This strategy does not take tardiness into account.
This could be a useful strategy for minimizing the total driven distance. 

Another strategy mainly based on location is the NearestDeliveryFristStrategy.
This is strategy is completly analogous to the NearestPickupFirstStrategy but ofcourse for deliveries the time windows are taken into account.
\\\\
The three other delivery strategies are more focussed on reducing time penalties.
The EarliestDeadlineStrategy tries to schedule deliveries as close to their time window.
The strategy first tries to reduce the earlieness to avoid long waits at the delivery location.
If by looking at earlieness there is no optimal parcel found the tardiness will be used to decide which parcel to deliver first. 

The MostTardyFirstStrategy first tries to deliver parcels that are already late or have the smallest window of delivery left. 
This could be a logical to ensure parcels are not starved.

NearestOnTimeDeliveryStrategy is the strategy we use for most of our experiments. 
Parcels that can be delivered in their time windows are the first candidates for delivery.
From those candidates the nearest delivery location is scheduled first.

\end{document}
